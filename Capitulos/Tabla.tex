\section{Tabla}

\parindent=0em
\pagenumbering{arabic}
\noindent

\begin{table}[H]
\centering
\begin{tabular}{|l|p{6cm}|p{4cm}|}% estableces número de columnas y posicion del texto
\hline % línea horizontal
\textbf{Tipo de síntesis} & \textbf{Descripción} & \textbf{Ejemplos} \\
\hline
\textbf{Sustractiva} & Elimina frecuencias mediante filtros. & Moog, Roland \\
\hline
\textbf{Aditiva} & Construye el sonido sumando múltiples ondas seno simples & Órganos electrónicos \\
\hline
\textbf{FM (frecuencia modulada)} & Modula la frecuencia de una onda con una moduladora & Yamaha DX7 \\
\hline
\textbf{Granular} & Reorganiza y procesa sonidos como conjuntos de pequeños fragmentos (granos) & Texturas, ambient \\
\hline
\textbf{Modelado físico} & Simula matemáticamente el comportamiento de un instrumento real  & Simuladores de instrumentos acústicos\\
\hline
\textbf{Síntesis espectral} & Manipula directamente el contenido frecuencial de un sonido & blabla \\
\hline
\end{tabular}
\caption{Comparativa de tipos de síntesis sonora}
\label{tab:sintesis}
\end{table}
